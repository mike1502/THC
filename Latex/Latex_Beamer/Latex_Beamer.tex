\documentclass[aspectratio=169, 10pt]{beamer}

\usepackage[utf8]{inputenc}
\usepackage[spanish,mexico]{babel}
\usepackage{amsmath, amsthm, amssymb}
\usepackage{hyperref}
\usepackage{graphicx}

\usepackage{float}
\usepackage{animate}
\usepackage{subcaption}
\usepackage{multicol}
\usepackage{verbatim}
\usepackage{color}
\usepackage{listings}
\usepackage{ragged2e}
\usepackage[spanish]{babel}
%\usepackage[latin1]{inputenc}
\usepackage{xcolor}


\usepackage{hyperref}
\hypersetup{
colorlinks=true, 
linkcolor=magenta, 
urlcolor=orange, 
citecolor=orange}

\urlstyle{same}
\usepackage[all]{xy}

\usetheme{Warsaw} % Warsaw, Bergen, Madrid, CambridgeUS, Berlin, Antibes 
\usecolortheme{seahorse} %albatross, beaver, crane, wolverine, seahorse
\title[\hspace{25mm} \insertframenumber/\inserttotalframenumber]{\bf Presentaciones con beamer}
%\subtitle{Clase 1}
%\author[Miguel Ángel Carrillo Lucía]{Ing. Miguel~Ángel~Carrillo~Lucía\inst{1,2}}

\author[Taller de Herramientas Computacionales - 2024-1] % (optional, for multiple authors)
{\hspace{2.70mm} \large{Miguel Ángel Carrillo Lucía} } %\inst{2}}

%% Institución
\institute[] % no es necesario llenar la opción
{
    %\inst{1}
    %Facultad de Ciencias\\
    Universidad Nacional Autónoma de México\\
    Facultad de Ciencias\\
    Departamento de Matemáticas\\
    Lic. en Matemáticas Aplicadas
}
%\institute{Ciudad de México}
\date{15 de agosto de 2025}

\usepackage{transparent}
\usepackage{eso-pic}

%\setbeamertemplate{background}{

%{\transparent{0.075}\includegraphics[width=1.0\paperwidth,height=1.0\paperheight]{escudos/matematicas_aplicadas.jpg}}
%}


\setbeamertemplate{background}{
  \rule{0pt}
  {.90\paperheight}%
  %{0.90}\paperheight
  \hspace*{.5\paperwidth}%
  \paperwidth1.50mm
  \makebox[1pt][c]{\transparent{0.120}\includegraphics[scale=0.56]{imagenes/logos_unam_ciencias.png} }
}

\begin{document}

%\maketitle

\begin{frame}
    \titlepage
    %\vspace{-50pt}
    %\begin{figure}[htpb]
    %    \begin{center}
    %        \includegraphics[width=0.145\linewidth]{escudos/escudo_ciencias_unam.png} \hspace{95mm}
    %        \includegraphics[width=0.145\linewidth]{escudos/escudo_unam.png}
    %    \end{center}
    %\end{figure}
\end{frame}

%\section{Agenda}
\begin{frame}{Agenda}
     \tableofcontents[sectionstyle=show,subsectionstyle=show/shaded/hide,subsubsectionstyle=show/shaded/hide]
\end{frame}



\section{La clase beamer}
\begin{frame}[fragile]{La clase beamer}
 \justifying
            \begin{block}{La clase beamer}
            \begin{itemize} \pause
            \justifying
                \item \verb|\documentclass[op1, op2]{beamer}|. Documento o clase de Latex diseñada para crear presentaciones de diapositivas profesionales. Se pueden crear múltiples diapositivas con su propio contenido (texto, imágenes, tablas, columnas, ecuaciones matemáticas, etc.). \pause
                \item Existe una gran variedad de temas y esquemas de colores predefinidos para personalizar las diapositivas. \pause
                \item La organización de la presentación se hace con secciones y subsecciones. \pause 
                \item Tiene formatos predefinidos para dar color a las diapositivas. \pause
                \item Se puede agregar título a la presentación, un subtítulo, añadir varios autores, incluir la fecha y logos de la institución o escuela.
            \end{itemize}
            \end{block} 

\end{frame}


\section{Preparar diapositiva}
\begin{frame}[fragile]{Preparar diapositiva} 
\begin{block}{Inicio} \pause
    \begin{enumerate}
        \item Abrir nuevo proyecto. Cambiar la opción documentclass a beamer \verb|\documentclass{beamer}|. \pause 
        \item Modificar título y añadir subtítulos. Cambiar la fuente del texto y el tamaño de letra. \pause
        \item Añadir autores con el comando \verb|\author{}|. Si son muchos, separar por comas (quedan de forma horizontal) o con salto de línea (\verb|\\|), para desplegarlos de forma vertical. Por último, modificar la fecha. 
    \end{enumerate}
\end{block}    
\end{frame}

\section{Añadir información de los ponentes}
\subsection{Añadir instituto y autores} \pause
\begin{frame}[fragile]{Preparar diapositiva}
\begin{block}{Añadir universidad, facultad, instituto} \pause

\verb|\institute[]| \pause % no es necesario llenar la opción \pause

\verb|{| \pause

    \verb|\inst{1}| \pause
    
    Facultad de Ciencias\\ \pause
    Universidad Nacional Autónoma de México \pause
    \\ 
    \verb|\and|

    \verb|\inst{2}| \pause
    
    Instituto de Ingeniería\\ \pause
    Universidad Nacional Autónoma de México \pause

\verb|}| 

\end{block}
\begin{alertblock}{Autores} \pause
    Se añade al autor \verb|\author{Miguel\inst{1,2}}|. 
\end{alertblock}
\end{frame}

\section{El entorno frame (diapositiva)}        
\begin{frame}[fragile]{Entorno frame (diapositiva)}
    \begin{block}{Iniciar diapositiva} \pause
        \begin{itemize} \pause
            \item Añadir frame/diapositiva con \verb|\begin{frame}[opciones]{Nombre}|. Añadir cuatro diapositivas más. \pause
        \end{itemize}
    \end{block}

    \begin{block}{Título de la presentación} \pause
        \verb|\begin{frame}[plain]| \pause
        
        \verb|\begin{center}| \pause
        
            \verb|\includegraphics[opciones]{ruta}| \pause
        
        \verb|\end{center}| \pause
        
        \verb|\titlepage| 
        
    \end{block}

    \begin{block}{Secciones} \pause
        Añadir secciones o subsecciones antes de cada frame/diapositiva. Serán útiles para cuando se defina el índice.
    \end{block}
    
\end{frame}

\subsection{Opciones de beamer}
\begin{frame}[fragile]{Opciones de beamer}
\begin{block}{Opciones de beamer} \pause
    \begin{itemize}
        \item Tamaño de letra (10 pt, 11pt, 12pt). \pause
        \item Relación de aspecto (aspectratio). Puede tomar los siguientes valores. \pause
        \begin{itemize}
            \item 169 $ \to $ 16:9. Modo panorámico. \pause
            \item 1610 $ \to $ 16:10 \pause
            \item 149 $\to $ 14:9 \pause
            \item 54 $\to$ 5:4 \pause
            \item 43 $\to $ 4:3. Estándar. \pause
            \item 32 $ \to $ 3:2
        \end{itemize} \pause
        \item \verb|\documentclass[aspectratio=169, 12pt]{beamer}|
    \end{itemize}
\end{block}
    
\end{frame}

\subsection{Definición del índice}
\begin{frame}[fragile]{Índice} 
    \begin{block}{Generar índice} \pause
        \verb|{| \pause
        
        \verb|\begin{frame}[t]| \pause
        
        \verb|\frametitle{índice general}| \pause
        
            \verb|\tableofcontents[currentsubsection]| \pause
        
        \verb|\end{frame}|    \pause
        
        \verb|}| 
        
    \end{block}
\end{frame}


\begin{frame}[fragile]{Mostrar índice cada cambio de sección o subsección}
    \begin{exampleblock}{Para secciones} \pause
    \verb|\AtBeginSection[ ]| \pause
    
        \verb|{|  \pause
        
        \verb|\begin{frame}<beamer>[allowframebreaks]| \pause

            \verb|\frametitle{Índice general}| \pause

        
                \verb|\tableofcontents[currentsection]| \pause
        
        \verb|\end{frame}| \pause
                
        \verb|}|
    \end{exampleblock}
    
\end{frame}


\begin{frame}[fragile]{Mostrar índice cada cambio de sección o subsección}
    \begin{exampleblock}{Para subsecciones} \pause
\verb|\AtBeginSubsection[ ]| \pause
    
        \verb|{| \pause
        
        \verb|\begin{frame}<beamer>[allowframebreaks]| \pause

            \verb|\frametitle{Índice general}| \pause

        
                \verb|\tableofcontents[currentsubsection]| \pause
        
        \verb|\end{frame}| \pause
                
        \verb|}| 
    \end{exampleblock}
    
\end{frame}

\section{Temas y colores}
\begin{frame}[fragile]{Temas y colores}
    \begin{block}{Temas y colores} \pause
        \begin{itemize}
            \item \verb|\usetheme{}|. Es el tema visual que se va a aplicar a las diapositivas. Están predefinidos en Latex en cuanto a diseño, colores, fuentes. Modifica el color de fondo, colores para el texto, diseño de encabezados y pie de página. \pause
            \item \verb|\usecolortheme{}|. Se utiliza para seleccionar el esquema de colores que se van a aplicar a las diapositivas. Ambos comandos vienen predefinidos en beamer. 
        \end{itemize}
        
    \end{block}
\end{frame}

\begin{frame}{Variedad de temas y colores}
	\begin{exampleblock}{Temas y colores}
		\url{https://deic.uab.cat/~iblanes/beamer_gallery/}
	\end{exampleblock}
\end{frame}

\section{Bloques y pausas}
\subsection{Bloques}
\begin{frame}[fragile]{Bloques y pausas} 
    \begin{alertblock}{Bloques y pausas} \pause
        \begin{itemize}
            \item Los bloques permiten añadir contenido (texto, listas, e incluso figuras) para resaltar la información que se presenta. Existen tres bloques básicos: \pause
            \begin{itemize}
                \item \verb|\begin{block}{Título}|. El color del bloque es muy suave. Suele ser de color azul* . Va a depender del tema a utilizar. Útil para presentar información que no es tan esencial. \pause
                \item \verb|\begin{exampleblock}{Título}|. El color del bloque suele ser verde* y se utiliza para mostrar ejemplos o casos de estudio específicos. \pause 
                \item \verb|\begin{alertblock}{Título}|. El bloque normalmente aparece en color rojo o un color fuerte. Se utiliza para presentar información importante, urgente o para llamar la atención de la audiencia.
            \end{itemize}
            
        \end{itemize}
        
    \end{alertblock}
\end{frame}


\begin{frame}{Bloques}
    \begin{block}{Bloque básico} \pause
        Contenido.
    \end{block}

    \begin{exampleblock}{Bloque para ejemplos} \pause
        Contenido.
    \end{exampleblock}

    \begin{alertblock}{Bloque de alerta} \pause
        Contenido.
    \end{alertblock}
\end{frame}

\subsection{Pausas}
\begin{frame}[fragile]{Pausas}
\begin{exampleblock}{Pausas} \pause
        \begin{itemize}
            \item El comando \verb|\pause| se utiliza para dividir una diapositiva en partes que se mostrarán en diferentes momentos de la presentación. \pause
            \item Va mostrando paso a paso (\textit{overlays}) el contenido de la diapositiva. Genera una nueva diapositiva por cada vez que se utilice. \pause
            \item Es muy útil para ir presentando de forma gradual el contenido de la exposición. \pause
        \end{itemize}

\end{exampleblock}

\end{frame}


\begin{frame}[fragile]{Ejemplo con pause}
    \begin{block}{Ejemplo con pause}
        \begin{itemize}
            \item Este es \pause
            \item un ejemplo \pause
            \item de cómo \pause
            \item se utiliza \pause
            \item el comando \pause
            \item \verb|\pause| 
        \end{itemize}
        
    \end{block}
    
\end{frame}


\begin{frame}[fragile]{Imagen de fondo y transparencia}
    \begin{exampleblock}{Transparencia}
        \begin{itemize}
            \item Para habilitar la opción de agregar una imagen (o texto) de fondo, se añade en el preámbulo el paquete \verb|\usepackage{transparent}|. 
            \item En el caso del texto, se puede utilizar el comando \verb|\transparent{N}|, donde N toma valores de 0 (opaco) a 1 (transparente).
            \item Ejemplo 
            \begin{itemize}
                \item \verb|\transparent{1}|: \transparent{1} El valor de transparent es 1 .
                \item \verb|\transparent{0.5}|: \transparent{0.5} El valor de transparent es 0.5 .
                \item \verb|\transparent{0.1}|: \transparent{0.1} El valor de transparent es 0.1 .
            \end{itemize}
        \end{itemize}
        
    \end{exampleblock}


\end{frame}


\begin{frame}[fragile]{Imagen de fondo}
    \begin{alertblock}{Imagen de fondo}
    Para añadir la imagen de fondo, se debe introducir el siguiente comando:
        \begin{itemize}
            \item \verb|\setbeamertemplate{background}|
            
            \verb|{|
            
            \verb|{\transparent{valor}\includegraphics|
            
            \verb|[width=1.0\paperwidth,height=1.0\paperheight]{ruta_imagen}|
            
            \verb|}|

            \item El valor máximo de width y height es 1. Inmediatamente después se añade el comando paperwidth o paperheight, respectivamente, para ajustar la imagen a la diapositiva.
        \end{itemize}
    \end{alertblock}
    
\end{frame}

\end{document}


