\documentclass[aspectratio=169, 10pt]{beamer}

\usepackage[utf8]{inputenc}
\usepackage[spanish,mexico]{babel}
\usepackage{amsmath, amsthm, amssymb}
\usepackage{hyperref}
\usepackage{graphicx}
%\newtheorem{theorem}{Teorema}[section]
\usepackage{tcolorbox}
\usepackage{float}
\usepackage{animate}
\usepackage{subcaption}
\usepackage{multicol}
\usepackage{verbatim}
\usepackage{color}
\usepackage{listings}
\usepackage{ragged2e}
\usepackage[spanish,mexico]{babel}
%\usepackage[latin1]{inputenc}
\usepackage{xcolor}
\usepackage{listings} %
\renewcommand{\lstlistingname}{Código}

%\lstset{language=python,
%    basicstyle=\ttfamily\normalsize,
%    keywordstyle=\color{blue},
%    stringstyle=\color{red},
%    commentstyle=\color{green},
%    showstringspaces=false,
%    numbers=left,
%    breaklines=true,
%    frame=single,
%    captionpos=b,
%}

\usepackage{hyperref}
\hypersetup{
colorlinks=true, 
linkcolor=magenta, 
urlcolor=orange, 
citecolor=orange}

\usepackage[all]{xy}

\usetheme{Warsaw} % Warsaw, Bergen, Madrid, CambridgeUS, Berlin, Antibes 
\usecolortheme{seahorse} %albatross, beaver, crane, wolverine, seahorse
\title[\hspace{25mm} \insertframenumber/\inserttotalframenumber]{\bf Introducción a Latex}
%\subtitle{Clase 1}
%\author[Miguel Ángel Carrillo Lucía]{Ing. Miguel~Ángel~Carrillo~Lucía\inst{1,2}}

\author[Taller de Herramientas Computacionales - 2024-1] % (optional, for multiple authors)
{\hspace{2.70mm} \Large{Miguel Ángel Carrillo Lucía}  \and \Large{Leonardo David Solís Rodríguez}} %\inst{2}}

%% Institución
\institute[] % no es necesario llenar la opción
{
    %\inst{1}
    %Facultad de Ciencias\\
    \large{Universidad Nacional Autónoma de México}\\
    \large{Facultad de Ciencias}\\
    \large{Departamento de Matemáticas}\\
    \large{Licenciatura en Matemáticas Aplicadas}
}
%\institute{Ciudad de México}
\date{\today}

\usepackage{transparent}
\usepackage{eso-pic}

%\setbeamertemplate{background}{

%{\transparent{0.075}\includegraphics[width=1.0\paperwidth,height=1.0\paperheight]{escudos/matematicas_aplicadas.jpg}}
%}


\setbeamertemplate{background}{
  \rule{0pt}
  {.90\paperheight}%
  %{0.90}\paperheight
  \hspace*{.5\paperwidth}%
  \paperwidth1.50mm
  \makebox[1pt][c]{\transparent{0.120}\includegraphics[scale=0.56]{imagenes/logos_unam_ciencias.png} }
}

\begin{document}

%\maketitle

\begin{frame}
    \titlepage
    %\vspace{-50pt}
    %\begin{figure}[htpb]
    %    \begin{center}
    %        \includegraphics[width=0.145\linewidth]{escudos/escudo_ciencias_unam.png} \hspace{95mm}
    %        \includegraphics[width=0.145\linewidth]{escudos/escudo_unam.png}
    %    \end{center}
    %\end{figure}
\end{frame}

%\section{Agenda}
\begin{frame}{Agenda}
     \tableofcontents[sectionstyle=show,subsectionstyle=show/shaded/hide,subsubsectionstyle=show/shaded/hide]
\end{frame}



\section{Ecuaciones}

\begin{frame}[fragile]{Ecuaciones}
    \begin{alertblock}{El entorno para ecuaciones.}
        \begin{itemize}
            \item Latex es muy útil para escribir ecuaciones. Para utilizarlo, es necesario añadir tres paquetes que a continuación se enlistan:
            \begin{enumerate}
                \item \textbf{amsmath} (AMS Math): Mejora el comportamiento y el aspecto de las ecuaciones. Permite añadir un asterisco en el entorno equation para crear ecuaciones sin enumerar.
                \item \textbf{amsthm} (AMS Theorem): define los entornos teorema, corolario, lema, definición.
                \item \textbf{amssymb} (AMS Symbol): carga otro paquete llamado amsfonts e incluye una colección de símbolos matemáticos.
            \end{enumerate}
            \item Los paquetes se pueden cargar por separado o bien en una sola línea de la siguiente forma:
            \begin{itemize}
                \item \verb|\usepackage{amsmath, amsthm, amssymb}|
            \end{itemize}
        \end{itemize}
    \end{alertblock}
\end{frame}

\subsection{Alineación de ecuaciones}
\begin{frame}{Alineación de ecuaciones}
    \begin{exampleblock}{Alineación de ecuaciones}
        \begin{itemize}
            \item La alineación de las ecuaciones se define en las opciones del tipo de documento (documentclass):
            \begin{enumerate}
                \item leqno: las ecuaciones están centradas pero la enumeración se coloca del lado izquierdo.
  
                \item fleqn: alinea las ecuaciones a la izquierda y las enumera del lado derecho.


                \item ceqn: centra las ecuaciones (opción por default si no se especifíca la alineación) y las enumera del lado derecho.
            \end{enumerate}
            
        \end{itemize}
    \end{exampleblock}

\begin{alertblock}{Cuidado.}
    \begin{itemize}
        \item Estas opciones son para todo el documento. No se pueden hacer combinaciones con leqno, fleqn, ceqn.
    \end{itemize}
    
\end{alertblock}
\end{frame}

\subsection{El modo matemático en Latex}
\begin{frame}[fragile]{El modo matemático en Latex}
    \begin{block}{Modo matemático en texto.}
        El modo matemático se puede poner de tres formas dentro del texto:
        \begin{enumerate}
            \item \verb|$ ecuación $| 
            \begin{itemize}
                \item Ejemplo: La variable \verb|$x^{2}$| está elevada al cuadrado. \hspace{1mm} $ \rightarrow $ \hspace{1mm} La variable $x^{2}$ está elevada al cuadrado.
            \end{itemize}
            \item \verb|[\ ecuación \]|
            \begin{itemize}
                \item Ejemplo: La variable \verb|[\ $x^{3} \]| está elevada al cubo \hspace{1mm} $ \rightarrow $ \hspace{1mm} La variable $x^{3}$ está elevada al cubo.
            \end{itemize}
            \item \verb|\begin{math} c^2=a^2+b^2 \end{math}| 
            \begin{itemize}
                \item Ejemplo: La variable \verb|\begin{math} x^3 \end{math}| está elevada al cubo \hspace{1mm} $\rightarrow$ \hspace{1mm} La variable \begin{math}
                    x^3 \end{math} está elevada al cubo
            \end{itemize}
        \end{enumerate}
    \end{block}

    $\ast$ \textit{No es necesario cargar paquetes}
    
\end{frame}

%\subsection{Matemática fuera del párrafo}
\begin{comment}
    

\begin{frame}[fragile]{Matemática fuera del párrafo}
\begin{columns}
    \begin{column}{.4\linewidth}
    \begin{block}{displaymath}
        \verb|\begin{displaymath}| \\
        \hspace{1cm}\verb|Ecuación| \\
        \verb|\end{displaymath}|   
    \end{block}
    \end{column}

    
    \begin{column}{.4\linewidth}
    \begin{exampleblock}{Ejemplo}
        La ecuación se coloca dentro del párrafo, por ejemplo: \begin{displaymath}
            c^2=a^2+b^2
        \end{displaymath} pero el comando de \textit{Overleaf} lo ordena en un renglón independiente.
    \end{exampleblock}
    \end{column}
\end{columns}

$\ast$ \textit{No es necesario cargar paquetes}
    
\end{frame}
\end{comment}
%\subsection{Enumerar ecuaciones}

\begin{frame}[fragile]{Ecuaciones. Numeradas y no enumeradas}

\textbf{Paquete:} \verb|\usepackage{amsmath}|
\vspace{0.3cm}

\begin{columns}
    \begin{column}{0.4\linewidth}
    \begin{block}{Entorno numerado}
        \verb|\begin{equation}|\\
        \hspace{1cm} \verb|c^2=a^2+b^2|\\
        \verb|\label{ec:Nombre}|
        \verb|\end{equation}|\\
    \end{block}
    \begin{block}{Entorno sin numerar}
        \verb|\begin{equation*}|\\
        \hspace{1cm} \verb|c^2=a^2+b^2|\\
        \verb|\end{equation*}|
    \end{block}
    \end{column}
    \begin{column}{0.4\linewidth}
    \begin{exampleblock}{Numerada}
        \begin{equation}
            c^2=a^2+b^2
            \label{ec:Pitagoras}
        \end{equation}
    \end{exampleblock}
    \vspace{1cm}
    \begin{exampleblock}{Sin Numerar}
        \begin{equation*}
            c^2=a^2+b^2
            \label{ec:Pitagoras}
        \end{equation*}
    \end{exampleblock}
    \end{column}
\end{columns}
    
\end{frame}

\subsection{Espacios, símbolos y operaciones}
\begin{frame}[fragile]{Espacios y símbolos}
    \begin{exampleblock}{Ejemplo}
        \verb|\begin{equation}|\\
        \hspace{1cm} \verb|x^2 \geq 0 \quad \text{para todo } x \in \mathbf{R}|\\
        \hspace{1cm} \verb|\label{ec:Ejemplo}|\\
        \verb|\end{equation}|
    \end{exampleblock}
    \begin{exampleblock}{Resultado}
        \begin{equation}
        x^2 \geq 0 \quad \text{para todo } x \in \mathbf{R}
        \label{ec:Ejemplo}
    \end{equation}
    \end{exampleblock}
   \textbf{Comandos:}
   \begin{table}[h!]
       \centering
       \begin{tabular}{c c | c c | c c }
           \verb|\quad| & Espacio normal &
           \verb|\,| & $3/18$ de \verb|\quad| &
           \verb|\:| & $4/18$ de \verb|\quad| \\
           \verb|\;| & $5/18$ de \verb|\quad| &
           \verb|\!| & $-3/18$ de \verb|\quad| &
           \verb|\qquad| & doble de \verb|\quad|\\
       \end{tabular}
       \caption{Espacios y símbolos}
       \label{tab:my_label}
   \end{table}

\end{frame}

\begin{frame}[fragile]{Algunos símbolos}

\begin{table}[]
    \centering
    \begin{tabular}{c c | c c | c c | c c}
        $\times$ & \verb|\times| & $\div$ & \verb|\div| & $\pm$ & \verb|\pm| & $\cdot$ & \verb|cdot| \\
        $\circ$ & \verb|\circ| & $\prime$ & \verb|\prime| & $\infty$ & \verb|\infty| & $\neg$ & \verb|neg| \\
        $\wedge$ & \verb|\wedge| & $\vee$ & \verb|\vee| & $\cup$ & \verb|\cup| & $\cap$ & \verb|cap| \\
        $\forall$ & \verb|\forall| & $\exists$ & \verb|\exists| & $\in$ & \verb|\in| & $\notin$ & \verb|notin| \\
        $\subset$ & \verb|\subset| & $\supset$ & \verb|\supset| & $\rightarrow$ & \verb|\rightarrow| & $\leftarrow$ & \verb|\leftarrow| \\
        $\leftrightarrow$ & \verb|\leftrightarrow| & $\Rightarrow$ & \verb|\Rightarrow| & $\Leftarrow$ & \verb|\Leftarrow| & $\Leftrightarrow$ & \verb|\Leftrightarrow| \\
        $\dot{a}$ & \verb|\dot{a}| & $\hat{a}$ & \verb|\hat{a}| & $\bar{a}$ & \verb|\bar{a}| & $\tilde{a}$ & \verb|\tilde{a}| \\
        $\neq$ & \verb|\neq| & $\geq$ & \verb|\geq| & $\leq$ & \verb|\leq| & $\sim$ & \verb|\sim| \\
        $\ldots$ & \verb|\ldots| & $\cdots$ & \verb|\cdots| & $\vdots$ & \verb|\vdots| & $\ddots$ & \verb|\ddots| \\
    \end{tabular}
    \caption{Símbolos}
    \label{tab:my_label}
\end{table}

\end{frame}

%\subsection{Operaciones}

\begin{frame}[fragile]{Operaciones}
    \begin{table}
        \centering
        \begin{tabular}{c c | c c | c c | c c}
             $\sqrt{a}$ & \verb|\sqrt{a}| & $\sqrt[n]{a}$ & \verb|\sqrt[n]{a}| & $\exp$ & \verb|\exp| & $\log$ & \verb|\log| \\ &&&&&&&\\%\vspace{5mm}
             $\ln$ & \verb|\ln| & $\sin$ & \verb|\sin| & $\cos$ & \verb|\cos| & $\tan$ & \verb|\tan| \\ &&&&&&& \\ %\vspace{5mm}
             $\lim_{a}$ & \verb|\lim_{a}| & $\sum_{a}^{b}$ & \verb|\sum_{a}^{b}| & $\prod_{a}^{b}$ & \verb|\prod_{a}^{b}| & $\int_{a}^{b}$ & \verb|\int_{a}^{b}| \\ &&&&&&& \\ %\vspace{5mm}
             $\iint$ & \verb|\iint| & $\oint$ & \verb|\oint| & $\partial$ & \verb|\partial| & $\nabla$ & \verb|\nabla| \\ &&&&&&& \\
             $\frac{a}{b}$ & \verb|\frac{a}{b}| & & & & & & \\
        \end{tabular}
        \caption{Operaciones}
        \label{tab:my_label}
    \end{table}
\end{frame}

\begin{frame}[fragile]{Ejercicio}
    \textbf{Escriba la fórmula general para resolver la ecuación de segundo grado}
    \begin{exampleblock}{Respuesta}
        \begin{equation}
                x_{1,2}=\frac{ -b\pm \sqrt{b^{2}-4ac}}{2a}
                \label{ec:Chicharronera}
        \end{equation}
    \end{exampleblock}
    La ecuación \ref{ec:Chicharronera} permite resolver ecuaciones de segundo grado.

    \vspace{5mm}
    $\ast$ Se hace referencia a la ecuación con el comando \verb|\ref{Nombre}|
    
\end{frame}

\subsection{Entornos}
\begin{frame}[fragile]{Entorno split}
\begin{alertblock}{Entorno split}
El entorno split sirve para dividir la ecuación en partes pequeñas o desarrollos de las mismas. Se utiliza el símbolo \& para definir a partir de dónde se fragmenta la expresión

    \verb|\begin{equation}|\\
    \hspace{1cm} \verb|\begin{split}|\\
    \hspace{2cm}\verb|5+7-2+4[12\cdot 3 -2]/2 & = 5+7-2+4[34]/2 \\| \\
    \hspace{2cm} \verb|& = 5+7-2+4[17]  \\| \\
    \hspace{2cm} \verb|& = 78| \\
    \hspace{1cm} \verb|\end{split}| \\
    \verb|\end{equation}|
\end{alertblock}


\end{frame}
\begin{frame}[fragile]{Entorno split}
    \begin{exampleblock}{Ejemplo}
    \begin{equation}
            \begin{split}
                5+7-2+4[12\cdot 3 -2]/2 & = 5+7-2+4[34]/2 \\
                       & = 5+7-2+4[17]  \\
                       & = 78
            \end{split}
    \end{equation}
\end{exampleblock}
\end{frame}


%\subsection{Entorno multline}
\begin{frame}[fragile]{Entorno multline}
\begin{block}{Ejemplo}
    \verb|\begin{multline}|\\
    \hspace{1cm}\verb|p(x) =-5x^{8} x^{7} - 3x^6 + 14x^5y + 590x^4y^2 + 19x^3y^3\\| \\
    \hspace{1cm} \verb|- 12x^2y^4 - 12xy^5 + 2y^6 - a^3b^3| \\
    \verb|\end{multline}|
\end{block}

\begin{exampleblock}{Resultado}
    \begin{multline}
            p(x) =-5x^{8} x^{7} - 3x^6 + 14x^5y + 590x^4y^2 + 19x^3y^3\\ 
            - 12x^2y^4 - 12xy^5 + 2y^6 - a^3b^3
        \end{multline}
\end{exampleblock}      
\end{frame}

%\subsection{Sistemas de ecuaciones}
\begin{frame}[fragile]{Sistemas de ecuaciones}
    \begin{columns}
        \begin{column}{0.4\linewidth}
        \begin{block}{Sin alinear}
            \verb|\begin{gather}| \\
            \hspace{1cm}\verb|2x + 2y =20 \\| \\
            \hspace{1cm}\verb|7x - 9y -2y = 213| \\
            \verb|\end{gather}| \\
        \end{block}
        \begin{block}{Alineada}
            \verb|\begin{align}| \\
            \hspace{1cm}\verb|2x + 2y &=20 \\| \\
            \hspace{1cm}\verb|7x - 9y -2y &= 213| \\
            \verb|\end{align}| \\
        \end{block}
        \end{column}
        \begin{column}{0.4\linewidth}
        \begin{exampleblock}{Resultado sin alinear}
            \begin{gather}
                2x + 2y =20 \\
                7x - 9y -2y = 213
            \end{gather}
        \end{exampleblock}
        \begin{exampleblock}{Resultado alineado}
            \begin{align}
                 2x + 2y &=20 \\
                7x - 9y -2y &= 213
            \end{align}
        \end{exampleblock}
        \end{column}
    \end{columns}
\end{frame}

%\subsection{Funciones a trozos}
\begin{frame}[fragile]{Funciones a trozos}
    \begin{columns}
        \begin{column}{0.5\linewidth}
        \begin{block}{Ejemplo}
            \verb|\begin{equation}| \\
            \hspace{0.5cm}\verb|f(x)= \left|\\ 
            \hspace{0.5cm}\verb|\{ \begin{array}{lcc}|\\
            \hspace{1cm}\verb|5 &   si  & x \leq 2 \\| \\
            \hspace{1cm}\verb| \\ x^2-6x+10 &  si & 2 < x < 5 \\|\\
            \hspace{1cm}\verb| \\ 4x-15 &  si  & x \geq 5|\\
            \hspace{0.5cm}\verb|\end{array}| \\
            \hspace{0.5cm}\verb|\right|\\
            \verb|\end{equation}|
        \end{block}
        \end{column}
        \begin{column}{0.5\linewidth}
        \begin{exampleblock}{Resultado}
            \begin{equation*}
                f(x)= \left 
                \{ \begin{array}{lcc}
                    5 &   si  & x \leq 2 \\
                    \\ x^2-6x+10 &  si & 2 < x < 5 \\
                    \\ 4x-15 &  si  & x \geq 5
                \end{array}
                \right.
            \end{equation*}
        \end{exampleblock}
        \end{column}
    \end{columns}
\end{frame}



\subsection{Entornos para teoremas y demostraciones.}
\begin{frame}[fragile]{Entornos para teoremas y demostraciones}

\begin{block}{Entorno Teorema Corolario, lema}

    \begin{itemize}
        \item \verb|\newtheorem{theorem}{Theorem}[section]|
        \begin{itemize}
            \item \verb|\theorem|. Es el entorno a iniciar.
            \item \verb|\Theorem|. Nombre del entorno (se puede poner en español).
            \item \verb|section| Para enumerar el teorema según la sección en que se esté posicionado.
        \end{itemize}
        \item \verb|\newtheorem{corollary}{Corollary}[theorem]|
        \item \verb|\newtheorem{lemma}[theorem]{Lemma}|
    \end{itemize}
\end{block}

\end{frame}


\begin{frame}{Teorema}
¿Qué nos dice el teorema fundamental del cálculo? Se menciona a continuación: 

\begin{theorem}
    El teorema fundamental del cálculo consiste en dos puntos: 
    \begin{enumerate}
        \item Si f es continua en [a,b], entonces la función 
        \begin{equation}
            g(x)=\int_{a}^{x}f(t)dt
        \end{equation} 
         es derivable en [a,b] y g'(x)=f(x) para cada x en [a,b].
        
        \item Si f es continua en [a,b] y si g es una función tal que g'(x)=f(x) para todo x en [a,b], entonces
        \begin{equation}
            \int_{a}^{b}f(x)dx=g(b)-g(a).
        \end{equation}
   \end{enumerate} 
\end{theorem}
\end{frame}

\subsubsection{Pero antes... tcolorbox}

\begin{frame}[fragile]{Entorno colorbox}

El entorno \verb|tcolorbox| permite crear cajas de texto con colores personalizados, sombras, bordes, opacidad entre otras características. El paquete asociado a este entorno es \verb|\usepackage{tcolorbox}|.

\begin{alertblock}{Entorno tcolorbox}
    El entorno se inicia de esta forma:
    
    \vspace{5mm}
    \verb|\begin{tcolorbox}[color de fondo, color del marco, título]|
    \vspace{5mm}

    \hspace{5mm} TEXTO, TEOREMA.

    \vspace{5mm}

    \verb|\end{tcolorbox}|
    
\end{alertblock}
\end{frame}


\begin{frame}[fragile]{Color de fondo, color del marco y título}


\begin{block}{Color de fondo y del marco}

    \begin{itemize}
        \item \verb|colframe=white|. Color del marco.
        \item \verb|colback=yellow|. Color de fondo.
        \item \verb|title=Título|. Título del teorema.
    \end{itemize}
    
\end{block}


\begin{alertblock}{Ejemplo}
    Entonces el entorno puede definirse así:


    \hspace{5mm} \verb|\begin{tcolorbox}[colframe=white, colback=yellow, title=Teorema]|.
\end{alertblock}

\end{frame}
% Configuración de listings en el preámbulo
%\section{Código Python}
%\begin{frame}[fragile]{Posters}
%\begin{lstlisting}[caption={Código Python}]

%import pandas as pd
%print("Impresion de una lista")
%array=[1,2,3,4,5]
%print(array)

%print("Creacion de una serie")
%miserie=pd.Series([10,11,12,13])
%print(miserie)
%print(miserie[1])

%#series con cadenas

%mistextos=pd.Series(["Miguel","Ivan","Daniel"])
%print(mistextos)
%\end{lstlisting}
%\end{frame}




\section{Código Python en \LaTeX}
\begin{frame}[fragile]{Código Python}
Para incluir código Python en Latex, se utiliza el paquete listings.

    \begin{enumerate}
        \item \verb|\usepackage{listings}| 
        \item \verb|\renewcommand{\lstlistingname}{Código}|. Este comando se utiliza para cambiar el nombre del código (parecido a los pies de figura).
    \end{enumerate}

Se deben hacer configuraciones adicionales para que Latex identifique los comentarios, palabras reservadas y cadenas de texto. Esto se hace con el comando \verb|\lstset|, el cuál requiere de parámetros adicionales para visualizar el código.
\end{frame}


\begin{frame}[fragile]{Código Python}

Los parámetros de \verb|\lstset| son:

    \begin{enumerate} \pause
        \item language=python. \textcolor{red}{Lenguaje de Programación.} \pause
        \item \verb|basicstyle|=\verb|\normalsize|. \textcolor{red}{Tipo y tamaño de letra.} \pause
        \item \verb|keywordstyle=\color{blue}|. \textcolor{red}{Resalta con color palabras reservadas de Python.} \pause
        \item \verb|stringstyle=\color{red}|, \textcolor{red}{Añade color a las cadenas de texto.} \pause
        \item \verb|commentstyle=\color{green}|. \textcolor{red}{Añade color a los comentarios.} \pause
    \end{enumerate}


\end{frame}


\begin{frame}[fragile]{Código Python}

Los parámetros de \verb|\lstset| son (continuación):

    \begin{enumerate} \pause
        \item \verb|showstringspaces=false|. \textcolor{red}{Evita colocar símbolos especiales entre espacios.} \pause
        \item \verb|numbers=left|. \textcolor{red} {Número de línea de código. Se muestra a la izquierda.} \pause
        \item \verb|breaklines=true|. \textcolor{red}{Mantiene el código dentro del marco (márgenes del documento).} \pause
        \item \verb|frame=single|. \textcolor{red}{Dibuja un marco sencillo.} \pause
        \item \verb|captionpos=b|. \textcolor{red}{Leyenda o 'pie de cuadro' en la parte de abajo.}
    \end{enumerate}


\end{frame}


\begin{frame}[fragile]{Código Python}

\begin{verbatim}
\begin{lstlisting}[caption={Primer código en Python.}]
		import pandas as pd
		print("Impresion de una lista")
		array=[1,2,3,4,5]
		print(array)

		#series con cadenas

		mistextos=pd.Series(["Miguel","Ivan","Daniel"])
		print(mistextos)    
\end{lstlisting}
\end{verbatim}

\end{frame}


\begin{frame}[fragile]{Código Python}

\begin{figure}
	\centering
	\includegraphics[scale=.525]{../../../../../Downloads/THC2026-1/codigopython2.png} 
\end{figure}
\end{frame}

\end{document}






